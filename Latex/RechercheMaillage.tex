\documentclass[10pt,a4paper]{article}

\usepackage[french]{babel}


\usepackage{amsmath}
\usepackage{amssymb}
\usepackage{graphicx}
%\usepackage{picins}
\usepackage{amsthm}
\usepackage{bbm}

\setlength{\voffset}{-28.4mm}
\setlength{\hoffset}{-1in}
\setlength{\topmargin}{20mm}
\setlength{\oddsidemargin}{25mm}
\setlength{\evensidemargin}{25mm}
\setlength{\textwidth}{160mm}

\setlength{\parindent}{0pt}

\setlength{\textheight}{235mm}
%\setlength{\footskip}{20mm}
\setlength{\headsep}{50pt}
\setlength{\headheight}{0pt}

\usepackage[french]{babel}

\begin{document}
	
\title{Recherche rapide d’un triangle contenant un point dans un maillage}

\section{}

R3: $<,>,<=,>=$\\
ordre lexicographie sur les pair est deja dans la stl

\subsection{Multimaps}
(Requires a less-than comparison function.)\\

$ O(3 n_T)$: jede Kante jedes Dreiecks als Schlüssel\\
$|K| \ O(log(n_T)):$ Zugriff auf Container fur jede Kante\\
$\Rightarrow O(n_T log(n_T))$\\
speichern in 2dim Array: Zugriff in konstanter Zeit\\

\textbf{unordered map}(hash tables) has constant time performance on all operations provided no collisions occur. When collisions occur, traversal of a linked list containing all elements of the same bucket (those that hash to the same value) is necessary, and in the worst case, there is only one bucket; hence O(n)\\

\subsection{Listen}

$ O(3 n_T)$, push front(): jede Kante jedes Dreiecks einfügen\\
$ O(N logN)$, container size $N$: sort(): definiere hierfür $<$ für R3 (Nach ersten und dann nach zweitem Element sortieren etc)\\

\textbf{doppelt verlinkt! und dann mit Kante und Sommet prev und next von Listenelement!}\\
konstanter Zugriff mit Iterator!\\

$ O(3 n_T)$: pop front und speichern in 2 dim array\\

Alternative für 2D array für konstanten Zugriff?\\
Daten könnten direkt in 2D array gespeichert werden, Listen und Maps überflüssig\\

\subsection{aire}

orientierter Flächeninhalt
$$ (a_i,b_i,p) = (a_i \times b_i ) \cdot p = \det(a_i,b_i,p) $$
bei det Vektoren in Zeilen

\section{Première tentative de documentation}

\subsection{Trouver le triangle adjacent}


\centering
 {\large \itshape{setAdjacencyViaMultimap} } 
  
\raggedright

Le but de cette méthode est l'initialisation des membres {\itshape neighbor1, neighbor2, neighbor3} de la classe triangle. Les membres sont décrits pas leurs position dans la liste des triangles. L'initialisation est réalisé en utilisant le container {\itshape multimaps}.  Pour chaque triangle $(a_1,a_2,a_3)$ on ajoute trois éléments au multimap où les arêtes $\{a_i,a_j \}, \, i,j \in \{1,2,3\}, \ i \neq j $ présentent les clés et le numéro de triangle est le valeur appliqué. Par conséquent, l'initialisation se produit dans $ O(3 n_T)$ et si deux triangles $(t_1,t_2) $ sont adjacents par l'arête $\{a,b\}$, les éléments $ ( \{a,b\}, t_1), \, \{a,b\}, t_2) $ ont les mêmes clés. 
Les triangles suivants sont trouvés par la procédure suivante: 
\begin{enumerate}
	\item On parcourt sur l'entière multimap en sauvegardant l'indice de  triangle $t_1 = (a_1,a_2,a_3) $ associé à l'arête $ \{a_i,a_j\} $ récente et écrasant l'élément $(\{a_i,a_j\}, t) $ de la map. La complexité est $ O(3 n_T)$.
	\item On cherche si le clé $ \{a_i,a_j\} $ est associé à un autre triangle $t_2$. Dans ce cas les triangles $t_1,t_2$ sont adjacents. 
	L'opération de recherche se passe selon la recherche dans un multimap dans $O(log N_T) $.
	\item On met $t_2 $ comme voisin $k \in \{1,2,3\}$ si le sommet de $t_1 $ au position $k$ n'est pas un sommet de $t_2 $ et vice versa. Cet opération est réalisé dans un temps constant. 
\end{enumerate}



\end{document}





